%!TEX encoding = UTF-8 Unicode
%!TEX TS-program = pdflatex
\documentclass[italian,a4paper,10pt]{report}
\usepackage{lmodern}
\usepackage[latin1]{inputenc}
\usepackage[italian]{babel}
\usepackage[T1]{fontenc}
\usepackage{float}
\usepackage{amsmath}
\usepackage{graphicx}
\usepackage{hyperref}

\usepackage{xcolor}
\usepackage{color}
\usepackage{tcolorbox}

\usepackage{titlesec}

\usepackage{pdfpages}


\usepackage{listings}% http://ctan.org/pkg/listings
\lstset{
  basicstyle=\ttfamily,
  mathescape
}

\begin{document}

\pagenumbering{gobble}
\includepdf[pages={1}]{fronte-frn.pdf}
\pagenumbering{arabic}
\setcounter{page}{1}

\paragraph*{Assunzioni}
In quanto non presente una vera e propria regola da utilizzare all'interno del ``prelude'', 
si � ricorsi alla funzione \texttt{if'} definita sulla wiki di haskell, 
presso il seguente indirizzo \href{https://wiki.haskell.org/If-then-else}{Wiki Haskell: If-Then-Else}

\subsection*{Esercizio 1}
Le regole sono quindi descritte come segue:
\begin{tcolorbox}[colback=white!5!white,colframe=cyan!75!black]
\begin{center}
\begin{verbatim}
        fact :: Integer -> Integer
R1:     fact n = if n == 0 then 1 else n * fact (n - 1)

        take :: Int -> [ a ] -> [ a ]
R2:     take _ [] = []
R3:     take n _ | n <= 0 = []
R4:     take n ( x : xs ) = x : take (n -1) xs

        map :: (a -> b) -> [a] -> [b]
R5:     map f ( x : xs ) = f x : map f xs
R6:     map _ [] = []

        if :: Bool -> a -> a -> a
R7:     if True  x _ = x
R8:     if False _ y = y
\end{verbatim}
\end{center}
\end{tcolorbox}

Si chiede di rappresentare l'esecuzione della seguente query:
\begin{tcolorbox}[colback=white!5!white,colframe=cyan!75!black,title=Query sottoposta]
\begin{center}
\begin{verbatim}
let 
    v = fact k;
    k = j - 2;
    j = 2 * 2 in 
  take (5 - k)(v : map fact [j - k, v, v, error "q",v])
\end{verbatim}
\end{center}
\end{tcolorbox}

\subsubsection*{Esecuzione in notazione lineare}
Per rendere univoca l'applicazione delle occorrenze dei metodi di classe, riscriviamo quindi la query come segue:
\begin{tcolorbox}[colback=white!5!white,colframe=cyan!75!black,]
\begin{center}
\begin{lstlisting}
take 
 (-#Int 5#Int $\kappa$)
 (
   $\nu$ : 
   map fact [-#Int $\iota$ $\kappa$, $\nu$, $nu$, error "q"#[Char], $\nu$]
 )
\end{lstlisting}
\end{center}
\end{tcolorbox}
Dove abbiamo:
\begin{itemize}
    \item $\nu = $\texttt{v}, con \texttt{v = fact} $\kappa$;
    \item $\kappa = $ \texttt{k}, con \texttt{k = -\#Int }$\iota$ \texttt{2\#Int};
    \item $\iota = $ \texttt{j}, con \texttt{j = *\#Int 2\#Int 2\#Int}.
\end{itemize}

Possiamo passare alla sua esecuzione.
~\\

[Valuto \texttt{R2}, ma non matcha.]

[Provo ad applicare \texttt{R3}, devo prima valutare il valore da passare alla guardia]
\begin{lstlisting}
(-#Int 5#Int $\kappa$)
\end{lstlisting}

[Dobbiamo valutare prima $\kappa$]\label{kappa}
\begin{lstlisting}
k = -\#Int $\iota$ 2\#Int
\end{lstlisting}

[Dobbiamo prima valutare $\iota$]\label{iota}\\
\lstinline{j = *#Int 2#Int 2#Int} $\rightarrow$ Viene valutato \texttt{4\#Int}.\\

[Torno a $\kappa$\ref{kappa}]\\
\lstinline{k = -#Int 4#Int 2#Int} $\rightarrow$ Viene valutato \texttt{2\#Int}.\\

[Torno a valutare il valore da passare alla guardia di \texttt{R3}]\\
\lstinline{-#Int 5#Int 2#Int} $\rightarrow$ Viene valutato \texttt{3\#Int}.\\

[Passo a valutare la guardia di \texttt{R3}]\\
\lstinline{<=#Int 2#Int 0#Int} $\rightarrow$ Viene valutato \texttt{False}.\\

[Procedo con la regola R4, matcha e produce]
\begin{lstlisting}
$\nu$ : take (-#Int 3#Int 1#Int) 
             (map fact [-#Int 4#Int 2#Int, $\nu$, $\nu$, error "q"#[Char], $\nu$])
\end{lstlisting}

[Devo ora valutare $\nu$]\label{nu}
\begin{lstlisting}
v = fact $\kappa$
\end{lstlisting}

[I passi per valutare $\kappa$ sono sopra\ref{kappa}]

[Applico \texttt{R1} per calcolare $\nu$]
\begin{lstlisting}
v = if (==#Int 2#Int 0#Int) then 
        1#Int 
        else (*#Int 2#Int (fact (-#Int 2#Int 1#Int)))
\end{lstlisting}

[Provo ad applicare \texttt{R7}, che mi obbliga a valutare la guardia dell'\texttt{if}]\\
\lstinline{==#Int 2#Int 0#Int} $\rightarrow$ Viene valutato \texttt{False}.\\

[\texttt{R7} fallisce, applico \texttt{R8} per la valutazione di $\nu$]
\begin{lstlisting}
v = (*#Int 2#Int (fact (-#Int 2#Int 1#Int)))
\end{lstlisting}

[Devo valutare l'operazione per la chiamata di \texttt{fact}]\\
\lstinline{-#Int 2#Int 1#Int} $\rightarrow$ Viene valutato \texttt{1\#Int}.\\

[Devo valutare la seconda chiamata a \texttt{fact} applicando \texttt{R1}]
\begin{lstlisting}
if (==#Int 1#Int 0#Int) then 
    1#Int 
    else (*#Int 1#Int (fact (-#Int 1#Int 1#Int)))
\end{lstlisting}

[Provo ad applicare \texttt{R7}, che mi obbliga a valutare la guardia dell'\texttt{if}]\\
\lstinline{==#Int 1#Int 0#Int} $\rightarrow$ Viene valutato \texttt{False}.\\

[\texttt{R7} fallisce, applico \texttt{R8} per la valutazione di $\nu$]
\begin{lstlisting}
v = (*#Int 2#Int (*#Int 1#Int (fact (-#Int 1#Int 1#Int))))
\end{lstlisting}

[Devo valutare il valore da passare a \texttt{fact}]\\
\lstinline{-#Int 1#Int 1#Int} $\rightarrow$ Viene valutato \texttt{0\#Int}.\\

[Devo valutare la terza chiamata a \texttt{fact} applicando \texttt{R1}]
\begin{lstlisting}
if (==#Int 0#Int 0#Int) then 
    1#Int 
    else (*#Int 0#Int (fact (-#Int 0#Int 1#Int)))
\end{lstlisting}

[Provo ad applicare \texttt{R7}, che mi obbliga a valutare la guardia dell'\texttt{if}]\\
\lstinline{==#Int 0#Int 0#Int} $\rightarrow$ Viene valutato \texttt{True}.\\

[Applico \texttt{R7} ed ottengo]
\begin{lstlisting}
v = (*#Int 2#Int (*#Int 1#Int (1#Int)))
\end{lstlisting}

[Devo valutare il primo fattore dell'operazione esterna \texttt{*\#Int}]\\
\lstinline{*#Int 1#Int 1#Int} $\rightarrow$ Viene valutato \texttt{1\#Int}.\\

[Devo valutare l'espressione da assegnare a $\nu$]\\
\lstinline{v = (*#Int 2#Int 1#Int)} $\rightarrow$ Viene valutato \texttt{2\#Int}.\\

[Al termine della valutazione di $\nu$ avr�]
\begin{lstlisting}
2#Int : 
  take (-#Int 3#Int 1#Int) 
    (map fact [-#Int 4#Int 2#Int, 2#Int, 2#Int, error "q"#[Char], 2#Int])
\end{lstlisting}

[Applico \texttt{R2} e non matcha, applico \texttt{R3} e devo valutare l'arg. di guardia]\\
\lstinline{-#Int 3#Int 1#Int} $\rightarrow$ Viene valutato \texttt{2\#Int}.

[Valuto quindi la guardia per \texttt{R3}]\\
\lstinline{<=#Int 2#Int 0#Int} $\rightarrow$ Viene valutato \texttt{False}.

[Applico quindi \texttt{R4}, devo valutare quindi \texttt{map} con regola \texttt{R5}]
\begin{lstlisting}
map fact [-#Int 4#Int 2#Int, 2#Int, 2#Int, error "q"#[Char], 2#Int]
\end{lstlisting}

[In uscita da \texttt{R5} avr�]
\begin{lstlisting}
(fact -#Int 4#Int 2#Int) : map fact [2#Int, 2#Int, error "q"#[Char], 2#Int]
\end{lstlisting}

[Posso quindi valutare \texttt{R4}]
\begin{lstlisting}
2#Int : 
  take (-#Int 2#Int 1#Int) 
    fact (-#Int 4#Int 2#Int) :
      map fact [2#Int, 2#Int, error "q"#[Char], 2#Int]
\end{lstlisting}

[che ottengo in uscita da \texttt{R4}]
\begin{lstlisting}
2#Int : fact (-#Int 4#Int 2#Int) : 
    take (-#Int 2#Int 1#Int) 
        map fact [2#Int, 2#Int, error "q"#[Char], 2#Int]
\end{lstlisting}

[Valuto il valore da passare a \texttt{fact}]\\
\lstinline{k = -#Int 4#Int 2#Int} $\rightarrow$ Viene valutato \texttt{2\#Int}.\\

[Si rimanda sopra per vedere la valutazione di \texttt{fact 2\#Int}]

[All'uscita di \texttt{fact 2} avr�]\\
\lstinline{fact 2#Int} $\rightarrow$ Viene valutato \texttt{2\#Int}.

[che diventa]
\begin{lstlisting}
 2#Int :2#Int :
    take (-#Int 2#Int 1#Int) 
        map fact [2#Int, 2#Int, error "q"#[Char], 2#Int]
\end{lstlisting}

[A questo punto, come prima si passa a rivalutare take.]\\

[Applico \texttt{R2} e non matcha, applico \texttt{R3} e devo valutare l'arg. di guardia]\\
\lstinline{-#Int 2#Int 1#Int} $\rightarrow$ Viene valutato \texttt{1\#Int}.

[Valuto quindi la guardia per \texttt{R3}]\\
\lstinline{<=#Int 1#Int 0#Int} $\rightarrow$ Viene valutato \texttt{False}.

[Applico quindi \texttt{R4}, devo valutare quindi \texttt{map} con regola \texttt{R5}]
\begin{lstlisting}
map fact [2#Int, 2#Int, error "q"#[Char], 2#Int]
\end{lstlisting}

[In uscita da \texttt{R5} avr�]
\begin{lstlisting}
2#Int : 2#Int : 
  take (-#Int 1#Int 1#Int) 
    fact 2#Int :
      map fact [2#Int, error "q"#[Char], 2#Int]
\end{lstlisting}

[Ed ottengo in uscita da \texttt{R4}]
\begin{lstlisting}
2#Int : 2#Int : fact 2#Int : 
    take (-#Int 1#Int 1#Int) 
        map fact [2#Int, error "q"#[Char], 2#Int]
\end{lstlisting}

[Si rimanda sopra per vedere la valutazione di \texttt{fact 2\#Int}]

[All'uscita di \texttt{fact 2} avr�]\\
\lstinline{fact 2#Int} $\rightarrow$ Viene valutato \texttt{2\#Int}.\\

[Otterr� quindi]
\begin{lstlisting}
2#Int : 2#Int : 2#Int : 
    take (-#Int 1#Int 1#Int) 
        map fact [2#Int, error "q"#[Char], 2#Int]
\end{lstlisting}

[Applico \texttt{R2} e non matcha, applico \texttt{R3} e devo valutare l'arg. di guardia]\\
\lstinline{-#Int 1#Int 1#Int} $\rightarrow$ Viene valutato \texttt{0\#Int}.\\

[Valuto quindi la guardia per \texttt{R3}]\\
\lstinline{<=#Int 0#Int 0#Int} $\rightarrow$ Viene valutato \texttt{True}.

[Applico \texttt{R3} dopo valutazione della guardia positiva]
\begin{lstlisting}
2#Int : 2#Int : 2#Int : []
\end{lstlisting}


\paragraph*{Risultato}
Il risultato della valutazione della query quindi �:
\begin{tcolorbox}[colback=white!5!white,colframe=cyan!75!black]
\begin{center}
\begin{verbatim}
2#Int : 2#Int : 2#Int : []
\end{verbatim}
\end{center}
\end{tcolorbox}

\subsection*{Esercizio 2}
\paragraph*{Esercizio 1 parte 2b.}
Di seguito la tabella riassuntiva di calcolo del testimone pi� generale dell'esercizio 1 parte 2b.
\begin{tcolorbox}[colback=white!5!white,colframe=lime!75!black]
\begin{center}
\begin{tabular}{cl}
mgu(\texttt{Ri, Rj}) & Testimone pi� generale \\
\hline
\hline
 \texttt{R2, R3} & \texttt{take n []} \\ 
 \texttt{R2, R4} & nessun overlap \\ 
 \texttt{R3, R4} & \texttt{take n (x : xs)} \\ 
 \texttt{R5, R6} & nessun overlap \\ 
 \texttt{R7, R8} & nessun overlap \\ 
\hline 
\end{tabular}
\end{center}
La regola \texttt{R1} non viene presa in considerazione in quanto non pu� avere overlap.
\end{tcolorbox}

\paragraph*{Esercizio 1 parte 2a.}
Di seguito presentiamo le regole dell'esercizio preso in considerazione:
\begin{tcolorbox}[colback=white!5!white,colframe=lime!75!black]
\begin{center}
\begin{verbatim}
fmap :: (a -> b) -> [a] -> [b]
R1:    fmap f (C x) = C $ f x
R2:    fmap f (Q ss sd is id) = Q (fmap f ss) (fmap f sd)
                                  (fmap f is) (fmap f id)

bound :: a -> a -> a -> a
R3:    bound cm cM a 
         | a > cM = cM
         | a < cm = cm
         | otherwise = a

boundPicture :: Integral t => t -> t -> QT t -> QT t
R4:    boundPicture cm cM a = fmap (bound cm cM) a
\end{verbatim}
\end{center}
\end{tcolorbox}

Di seguito la tabella riassuntiva di calcolo del testimone pi� generale per questo esercizio.
\begin{tcolorbox}[colback=white!5!white,colframe=lime!75!black]
\begin{center}
\begin{tabular}{cl}
mgu(\texttt{Ri, Rj}) & Testimone pi� generale \\
\hline
\hline
 \texttt{R1, R2} & nessun overlap \\
\hline 
\end{tabular}
\end{center}
Le regole \texttt{R3} ed \texttt{R4} non vengono prese in considerazione in quanto non possono avere overlap.
\end{tcolorbox}
\end{document}