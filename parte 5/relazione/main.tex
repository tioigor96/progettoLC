\documentclass{article}
\usepackage[utf8]{inputenc}
\usepackage{bussproofs}
\title{ProgettoLC parte5parz Gruppo 2 Relazione}
\author{Gruppo 2 }

\usepackage{tikz}
\usetikzlibrary{arrows.meta,positioning,matrix}

\usepackage{listings}

\begin{document}

\maketitle

\section{Scelte implementative}
L'implementazione del auL, seppur semplice, denota alcune particolarità, dovuta, tra le altre cose, a voler utilizzare
meno attributi possibili in Happy, e di creare un linguaggio solido.

La valutazione della dereferenziazione viene effettuata leggendo la parte di AST creata, dando precedenza all'array
e successivamente al puntatore, in quanto il tipo complesso (ad esempio, *Int[]) viene memorizzato come array di
puntatori.

Viene creato e manipolato passo-passo un SDT: questo mi permette di creare quindi sia un AST che un TAC contemporaneamente.

In auL sono presenti le modalità di passaggio dei parametri by value (val), by reference (ref), by constant (const), by 
result (res) and by value/result (valres). Se la modalità non è definita, verrà usata la modalità di default che è by 
reference per array e stringhe e by value per tutti gli altri tipi.

Un programma auL è composto da una lista di Statement e/o delle funzioni: il file per essere accettato deve contenere
almeno una delle due. All'interno del blocco delle funzioni, invece, similmente ai blocchi degli statement, non è richiesto
sia presente alcuna riga di codice.

\section{Lessico di auL}
in controllo sintattico viene lasciato completamente al compilatore. Noi abbiamo sviluppato solamente la gestione
degli errori sintattici. si definisce quindi che:
\begin{itemize}
	\item un'etichetta auL sia quindi definita come\\
		\texttt{token LIdent (letter|'\_'lower)(letter|digit|'\_')*;};
	
	\item i letterali che rappresentano valori costanti di tipo intero, floating point, carattere e stringhe 
	seguono le normali convenzioni, i letterali false e true rappresentano i booleani, \texttt{'nil'} per il valore
	dei puntatori.
\end{itemize}
\subsection*{Parole riservate}
Le parole riservate in auL sono:
\begin{verbatim}
  Bool       Char       Float       Int
  String     Void       and         break
  const      do         else        elseif
  end        false      for         function
  if         in         local       name
  nil        not        or          readChar
  readFloat  readInt    readString  ref
  repeat     res        return      then
  true       until      val         valres
  while      writeChar  writeFloat  writeInt
  writeString
\end{verbatim}

\subsection*{Simboli riservati}
I simboli riservati sono:
\begin{verbatim}
  | * | [ | ] | ;
  | = | { | } | ,
  | ( | ) | == | ~=
  | < | <= | > | >=
  | + | - | / | %
  | ^ | .. | # | &
\end{verbatim}
\subsection*{Commenti}
Sono presenti sia commenti in-line che commenti multi-line. Sono definiti quindi dalle seguenti espressioni regolari
\begin{verbatim}
-- Toss single line comments
"--" [.]* ;
-- Toss multi line comments
"--[" ("=")* "[" ([$u # [\]]])* ("]") ("=")* "]" ;
\end{verbatim}

\section{Scoping e binding  di auL}
Lo scoping di auL è prettamente statico, similmente a C. È possibile definire quindi sole funzioni ricorsive semplici
e non mutuamente ricorsive. Lo scoping del linguaggio, essendo compilato, è statico.

La visibilità dei nomi, siano queste di funzioni o di variabili, è immediatamente successivo alla loro dichiarazione 
ed a tutti i loro blocchi innestati (a meno di redifinizione nei suoi blocchi). È possibile bypassare la definizione 
di un qualunque nome attraverso la regola del \textit{local}: local infatti permette di \textbf{definire una variabile} 
riutilizzando un nome già utilizazato, anche nel medesimo blocco.

Non è possibile dichiarare funzioni innestate: quando si entra in un blocco legato ad uno statement, si perde
la possibilitò di definirne.

Non è possibile definire variabili locali come costanti, ma è possibile definire parametri di funzione
come tali. 

La definizione di variabili (ed array o puntatori) permette la loro inizializzazione: limite dell'inizializzazione è
che non possono autoriferirsi durante la loro definizione (a meno non ci si trovi in una situazione di utilizzo dello
statement \textit{local}). La loro non inizializzazione crea un valore generico al suo interno (undefined behaviour),
ed in sè non genera un errore. Non si richiede che il valore di inizializzazione sia noto a \textit{compile time}.

\section{Tipi e regole di auL}
\subsection{Definizione dei tipi}
In auL abbiamo identificato i seguenti tipi di base
\begin{itemize}
    \item Int
    \item Char
    \item String
    \item Float
    \item Boolean
\end{itemize}
Inoltre c'è la possibilità di definire array di dimensione prefissata e di definire dei puntatori a tipo qualsiasi. Questi sono quindi noti come
``higher-types''.

Abbiamo anche definito il tipo Void per gestire il tipo di ritorno delle funzioni che non restituiscono alcun valore.
\subsection{Compatibiltà tra i tipi}
Definiamo come \emph{tipi canonici} i tipi $C_\tau=\{Int, Float, Bool, Char, String\}$.


Mostriamo quindi la gerarchia di tipo:\\
\begin{tikzpicture}
\matrix (m) [matrix of math nodes,
row sep=3em, column sep=3em,
text height=1.5ex,
text depth=0.25ex]{
            &            &              &                             \\
Float       &   String   &              &                             \\
            &            &              &                             \\
Int         &  Char      &  Bool        &       Void                  \\
};
\path[-{Latex[length=2.5mm, width=1.5mm]}]
(m-4-1) edge (m-2-1)
(m-4-2) edge (m-2-2);
\end{tikzpicture}
~\\
~\\

Quindi deduciamo che:
\begin{itemize}
	\item se assegniamo un \emph{Int} ad un \emph{Float}, questo sarà accettato;
	\item se ``\emph{dereferenziamo}'' un tipo \emph{String} attraverso come ad un array,
		questo accetterà un carattere;
	\item ogni $C_{\tau}$ accetterà ogni $C_{\tau}$;
	\item se $\tau_1$ è un \emph{higher-type} puntatore, questo accetterà un $\tau_2$ \emph{higher-type} array, a patto che
		l'``\emph{altezza}'' del puntatore sia la medesima dell'``\emph{altezza}'' e\\ $C_{\tau_1} = C_{\tau_2}$.
	
\end{itemize}

\section{Type system}
\subsection{Regole per le costanti}
Sia ``$a$'' un valore per un tipo ``$\tau$". Allora
	\begin{prooftree}
		\AxiomC{$\exists a \in \{\tau:Int,\tau:Bool,\tau:Float,\tau:Char,\tau:String\}$}
		\UnaryInfC{$\Gamma \vdash_{val} a : \tau$}
	\end{prooftree}

Esemplifichiamo ulteriormente ``$a$'' un valore per un tipo ``$\tau:Bool$":
	\begin{prooftree}
		\AxiomC{$a =$ `$true$'$: \tau$}
		\AxiomC{$\tau:Bool$}
		\BinaryInfC{$\Gamma \vdash_{val} a : \tau$}
	\end{prooftree}
	\begin{prooftree}
		\AxiomC{$a =$ `$false$'$: \tau$}
		\AxiomC{$\tau:Bool$}
		\BinaryInfC{$\Gamma \vdash_{val} a : \tau$}
	\end{prooftree}
Esemplifichiamo ulteriormente ``$a$'' un valore per un tipo puntatore a ``$\tau:C_\tau$":
	\begin{prooftree}
		\AxiomC{$a = $ `$nil$'$: \tau$}
		\AxiomC{$\tau: Pointer(C_\tau)$}
		\BinaryInfC{$\Gamma \vdash_{val} a : Pointer(C_\tau)$}
	\end{prooftree}
\subsection{Regole per le Right-Expressions}
\textit{S.p.d.g.} siano $a$ e $b$ delle \emph{Right expression}. Esemplifichiamo quindi il type system per le \emph{Right Expressions}.
\begin{itemize}
	\item $\tau\in\{Int,Float\}$
	\begin{prooftree}
		\AxiomC{$\Gamma \vdash_{rexp} a : \tau$}
		\AxiomC{$ -a : \tau$}
		\BinaryInfC{$\Gamma \vdash_{rexp} - a : \tau$}
	\end{prooftree}

	\item $\tau_1,\tau_2\in\{Int,Float,Char\}$, $\tau_\diamond$ rispetta la gerarchia di tipo e\\ $\odot \in \{+,-\}$
	\begin{prooftree}
		\AxiomC{$\Gamma \vdash_{rexp} a : \tau_1$}
		\AxiomC{$\Gamma \vdash_{rexp} b : \tau_2$}
		\AxiomC{$a \odot b : \tau_\diamond$}
		\TrinaryInfC{$\Gamma \vdash_{rexp} a \odot b : \tau_\diamond$}
	\end{prooftree}

	\item $\tau_1,\tau_2\in\{Int,Float\}$, $\tau_\diamond$ rispetta la gerarchia di tipo e\\ $\odot \in \{*, \div, \hat{\mkern6mu} \}$
	\begin{prooftree}
		\AxiomC{$\Gamma \vdash_{rexp} a : \tau_1$}
		\AxiomC{$\Gamma \vdash_{rexp} b : \tau_2$}
		\AxiomC{$a \odot b : \tau_\diamond$}
		\TrinaryInfC{$\Gamma \vdash_{rexp} a \odot b : \tau_\diamond$}
	\end{prooftree}

	\item $\tau_1,\tau_2\in\{Int,Float\}$
	\begin{prooftree}
		\AxiomC{$\Gamma \vdash_{rexp} a : \tau_1$}
		\AxiomC{$\Gamma \vdash_{rexp} b : \tau_2$}
		\AxiomC{$a \% b : Int$}
		\TrinaryInfC{$\Gamma \vdash_{rexp} a \% b : Int$}
	\end{prooftree}

	\item $\tau_1,\tau_2\in\{Char,String\}$
	\begin{prooftree}
		\AxiomC{$\Gamma \vdash_{rexp} a : \tau_1$}
		\AxiomC{$\Gamma \vdash_{rexp} b : \tau_2$}
		\AxiomC{$a .. b : String$}
		\TrinaryInfC{$\Gamma \vdash_{rexp} a .. b : String$}
	\end{prooftree}

	\item $\tau_1,\tau_2 = C_\tau$ e $\odot \in \{\leq,\geq,<,>,==,\not=\}$
	\begin{prooftree}
		\AxiomC{$\Gamma \vdash_{rexp} a : \tau_1$}
		\AxiomC{$\Gamma \vdash_{rexp} b : \tau_2$}
		\AxiomC{$a \odot b : Bool$}
		\TrinaryInfC{$\Gamma \vdash_{rexp} a \odot b : Bool$}
	\end{prooftree}
	
	\item dove $\tau$ è puntatore a $C_\tau$ e $\odot \in \{==,\not=\}$
	\begin{prooftree}
		\AxiomC{$\Gamma \vdash_{val} a :$ `$nil$'}
		\AxiomC{$\Gamma \vdash_{rexp} b : \tau$}
		\AxiomC{$a \odot b : Bool$}
		\TrinaryInfC{$\Gamma \vdash_{rexp} a \odot b : Bool$}
	\end{prooftree}
	
	\item  il caso riflessivo per $a$ e $b$ a quello sopra;

	\item $\tau_1, \tau_2$ sono puntatore a $C_\tau$ e $\odot \in \{==,\not=\}$
	\begin{prooftree}
		\AxiomC{$\Gamma \vdash_{rexp}a : \tau_1$}
		\AxiomC{$\Gamma \vdash_{rexp}b : \tau_2$}
		\AxiomC{$a \odot b : Bool$}
		\TrinaryInfC{$\Gamma \vdash_{rexp} a \odot b : Bool$}
	\end{prooftree}

	\item $\tau$ è puntatore a $C_\tau$
	\begin{prooftree}
		\AxiomC{$\Gamma \vdash_{rexp}a : \tau$}
		\AxiomC{$\#a : Int$}
		\BinaryInfC{$\Gamma \vdash_{rexp} \#a : Int$}
	\end{prooftree}

	\item $\odot \in \{and, or\}$
	\begin{prooftree}
		\AxiomC{$\Gamma \vdash_{rexp}a : Bool$}
		\AxiomC{$\Gamma \vdash_{rexp}b : Bool$}
		\AxiomC{$a \odot b : Bool$}
		\TrinaryInfC{$\Gamma \vdash_{rexp} a \odot b : Bool$}
	\end{prooftree}
	
	\item regola unaria del not
	\begin{prooftree}
		\AxiomC{$\Gamma \vdash_{rexp}a : Bool$}
		\AxiomC{$not~a : Bool$}
		\BinaryInfC{$\Gamma \vdash_{rexp}not~a : Bool$}
	\end{prooftree}
	
	\item siano $a_1,\dots,a_n$ espressioni, $id$ identificativo di una funzione con arietà $n$ e $\tau_\diamond\in C_\tau$
	\begin{prooftree}
		\AxiomC{$\Gamma \vdash_{rexp}a_1:\tau_1,\dots,\Gamma \vdash_{rexp}a_n:\tau_n$}
		\AxiomC{$\Gamma \vdash_{rexp}id(a_1:\tau_1 \times \dots \times a_n:\tau_n) : \tau_\diamond$}
		\BinaryInfC{$\Gamma \vdash_{rexp}id(a_1,\dots,a_n) : \tau_\diamond$}
	\end{prooftree}

	\item $id$ identificativo di una variabile, $\tau$ il suo tipo (sia higher-type che $C_\tau$)
	\begin{prooftree}
		\AxiomC{$\Gamma \vdash_{lexp}id:\tau$}
		\UnaryInfC{$\Gamma \vdash_{rexp}id : \tau$}
	\end{prooftree}

	\item $id$ identificativo di una variabile, $\tau$ il suo tipo (sia higher-type che $C_\tau$) e $*\tau$ il puntatore a $\tau$
	\begin{prooftree}
		\AxiomC{$\Gamma \vdash_{lexp}id:\tau$}
		\UnaryInfC{$\Gamma \vdash_{rexp}\&id : *\tau$}
	\end{prooftree}
	
	\item sia $\phi\in\{readInt():Int,readFloat():Float,readChar():Char,readString():String\}$ funzioni del linguaggio 
		per ricevere input dal mondo esterno, ed associamo loro il tipo di ritorno
	\begin{prooftree}
		\AxiomC{$\Gamma \vdash_{rexp}\phi():\tau$}
		\UnaryInfC{$\Gamma \vdash_{rexp}\phi() : \tau$}
	\end{prooftree}
	
	\item $val$ identificativo di un valore, $\tau$ il suo tipo (sia higher-type che $C_\tau$)
	\begin{prooftree}
		\AxiomC{$\Gamma \vdash_{val}val:\tau$}
		\UnaryInfC{$\Gamma \vdash_{rexp}val : \tau$}
	\end{prooftree}
\end{itemize}
\subsection{Regole per le Left-Expression}
\textit{S.p.d.g.} sia $a$ una Left Expression. Esemplifichiamo il type system per le Left Expression.

\begin{itemize}	
	\item sia $*\tau : Pointer(C_\tau)$
	\begin{prooftree}
		\AxiomC{$\Gamma\vdash_{lexp} a : *\tau$}
		\UnaryInfC{$\Gamma \vdash_{lexp} *a : \tau$}
	\end{prooftree}
	
	\item sia $\tau[~] : Array(C_\tau)$, e sia $E \dashv_{rexp} \Gamma : Int$
	\begin{prooftree}
		\AxiomC{$\Gamma\vdash_{lexp} a : \tau[~]$}
		\AxiomC{$\Gamma\vdash_{rexp} E : Int$}
		\BinaryInfC{$\Gamma \vdash_{lexp} a[E] : \tau$}
	\end{prooftree}
	\item sia $*\tau : Poiter(C_\tau)$, e sia $E \dashv_{rexp} \Gamma : Int$
	\begin{prooftree}
		\AxiomC{$\Gamma\vdash_{lexp} a : *\tau$}
		\AxiomC{$\Gamma\vdash_{rexp} E : Int$}
		\BinaryInfC{$\Gamma \vdash_{lexp} a[E] : \tau$}
	\end{prooftree}
\end{itemize}
\subsection{Regole per la definizione di funzioni}
\textit{S.p.d.g.} sia $C$ un insieme (anche vuoto) di Statement. Esemplifichiamo il type system per la definizione di funzioni.
\begin{itemize}
	\item sia $\tau\in\{HT(C_\tau), C_\tau \cup \{Void\}\}$, $\Gamma$ l'\emph{Environment} e $\hat{\Gamma}$ l'\emph{environment} nuovo
	\begin{prooftree}
		\AxiomC{$\Gamma, id\vdash \emptyset :\perp$\footnote{~}} 
		\AxiomC{$\Gamma \vdash_{cmd} C:Void$}
		\BinaryInfC{$\Gamma \vdash_{fnct} id(a_1:\tau_1,\dots,a_n:\tau_n)\{C\} : \tau \Rightarrow (\Gamma,id(a_1:\tau_1 \times a_n:\tau_n):\tau)\models\hat{\Gamma}$ }
	\end{prooftree}
	\begin{prooftree}
		\AxiomC{$\Gamma, id\vdash \emptyset :\perp$}
		\AxiomC{$\Gamma \vdash_{cmd} C:Void$}
		\BinaryInfC{$\Gamma \vdash_{fnct} id(~)\{C\} : \tau \Rightarrow (\Gamma,id(~):\tau)\models\hat{\Gamma}$ }
	\end{prooftree}
	\begin{enumerate}
	\item con ``$\Gamma, id\vdash \emptyset :\perp$'' si indica che non è mai stata definita nessuna funzione o variabile con nome definito in ``id''.
	\end{enumerate}
\end{itemize}
\subsection{Regole per la definizione di variabili}
\textit{S.p.d.g.} sia $\Gamma$ l'\emph{Environment} e $\hat{\Gamma}$ l'\emph{environment} nuovo.
Esemplifichiamo il type system per la definizione di funzioni.
\begin{itemize}
	\item sia $\tau\in\{HT(C_\tau), C_\tau\}$
	\begin{prooftree}
		\AxiomC{$\Gamma, id\vdash \emptyset :\perp$}
		\UnaryInfC{$\Gamma \vdash_{var} id : \tau \Rightarrow (\Gamma,id:\tau)\models\hat{\Gamma}$ }
	\end{prooftree}
	\begin{prooftree}
		\AxiomC{$\Gamma, id\vdash \emptyset :\perp$} 
		\AxiomC{$\Gamma \vdash_{rexp} b : Int$}
		\BinaryInfC{$\Gamma \vdash_{var} id[b] : \tau \Rightarrow (\Gamma,id:\tau)\models\hat{\Gamma}$ }
	\end{prooftree}	
	
	\item sia $\tau\in\{Pointer(C_\tau), C_\tau\}$
	\begin{prooftree}
		\AxiomC{$\Gamma, id\vdash \emptyset :\perp$} 
		\AxiomC{$\Gamma \vdash_{rexp} E :\tau$}
		\BinaryInfC{$\Gamma \vdash_{var} id = E : \tau \Rightarrow (\Gamma,id:\tau)\models\hat{\Gamma}$ }
	\end{prooftree}
	
	\item sia $\tau : HT(C_\tau)$
	\begin{prooftree}
		\AxiomC{$\Gamma \vdash_{val} a_i :\tau , \forall i \in [1,n]$}
		\AxiomC{$\Gamma \vdash_{val} E=\{a_1:\tau \times \dots \times a_n:\tau\} : \tau $}
		\BinaryInfC{$\Gamma \vdash_{val} E : \tau$}
		\AxiomC{$\Gamma, id\vdash \emptyset :\perp$} 
		\AxiomC{$\Gamma \vdash_{rexp} b : Int$}
		\TrinaryInfC{$\Gamma \vdash_{var} id[b] = E : \tau \Rightarrow (\Gamma,id:\tau)\models\hat{\Gamma}$ }
	\end{prooftree}
\end{itemize}

\subsection{Regole per i comandi}
\emph{S.p.d.g.}, per tutti i comandi appartenenti a ``\textit{auL}'', possiamo affermare che il tipo che ritorna sia sempre e solo indefinito.
\begin{prooftree}
	\AxiomC{}
	\UnaryInfC{$\Gamma\vdash_{cmd}C:\perp$}
\end{prooftree}
\section{Altri vincoli del auL}
\begin{itemize}
	\item Una chiamata di funzione in una espressione non può avere tipo di
		ritorno void, perché le espressioni non possono avere tipo void;
	
	\item \textbf{break} e \textbf{continue} possono comparire solo nel corpo di un loop.
		In caso di loop annidati, questi statement si riferiscono al ciclo più interno;
	
	\item poichè il return è prerogativa nelle sole funzioni, e non è possibile verificare che
		venga eseguito ogni return dichiarato nei blocchi (come ``\textit{if-then-elseif-else}'',
		si è deciso che sia sufficiente lasciare che c'è almeno un blocco che lo contiene perchè
		la funzione di tipo non-\textit{Void} abbia il return richiesto. Rimane a carico del programmatore
		garantirne il corretto funzionamento;

	\item nel caso in cui il tipo della funzione sia definito come \textit{Void}, si controlla solo che
		il \textit{return} non ritorni effettivamente nulla. Oltre questo, viene considerato valido se
		in questo genere di funzioni non compaia il return.
	
	\item viene controllato, ad ogni chiamata di funzione, che il numero di parametri
		attuali coincida con il numero di parametri formali, e che i tipi
		siano compatibili;
		
	\item non è possibile sovrascrivere una costante, perciò negli assegnamenti
		si controlla che la lhs non sia una costante;
	
	\item **da mettere la roba del name/const/...**.
\end{itemize}


\section{TAC}
Blablablablablablablablablabla...

\section{Implementazione di auL}
Dapprima si è iniziato a definire una grammatica completa di tutte le feature desiderate da auL attraverso l'ausilio di
BNFC. Successivamente, dopo aver migrato la grammatica di base di Happy generata da BNFC ad una attributata,
si è iniziato a definire l'environment (\textit{Env.hs}). Finito di sviluppare le regole base per le dichiarazioni 
di tipi canonici e le right-expression, si è implementata in maniera iterativa tutte le altre funzionalità pensate 
nel linguaggio.

La grammatica attributata si compone di:
\begin{itemize}
	\item alcuni attributi per la generazione dell' AST;
	\item alcuni attributi per la gestione dell'environment, tra cui il tipo uscente dalle RExp, l'env.
		locale, env. esterno e env. uscente (ovvero le eventuali modifiche dell'environment locale). Il
		controllo del tipo e della presenza del return in una funzione viene effettuato attraverso la manipolazione
		dell'env. loc del blocco-funzione, dove si controlla se effettivamente viene trovato o meno, e se è del tipo 
		desiderato.
	\item gli attributi per il TAC **blablablabla**
\end{itemize}

Parallelamente, dopo aver sviluppato l'environment, si è iniziato a scrivere il file per il TAC e la sua 
gestione.

\section{Conflitti}
Non sono presenti conflitti nella grammatica definita da BNFC.
\end{document}
